\documentclass{beamer}

\title{Analyst’s Traveling Salesman Problem Algorithm Implementation}
\author{Gabriel Johnson and Robby Lawrence}
\date{October 17, 2024}

\begin{document}

\frame{\titlepage}

\begin{frame}
\frametitle{What is the Analyst’s Traveling Salesman Problem (ATSP)?}
\begin{itemize}
    \item The ASTP is a problem related to the more well-known
    Traveling Salesman Problem (TSP).
    \item While the TSP aims to find the shortest path touring a finite set of points, the ATSP asks if there exists a curve of finite length (rectifiable) that contains a given set of points in $\mathbb{R}^n$.
    \item We follow the algorithm laid out in Time Complexity of the Analyst Traveling Salesman written by Anthony Ramirez and Vyron Vellis.
\end{itemize}
\end{frame}

\begin{frame}
\frametitle{Language and Libraries}
    \begin{itemize}
        \item Our code is written in C++
        \item We use igraph for efficient graph manipulation
        \item Building is handled with CMake
    \end{itemize}
\end{frame}

\begin{frame}
\frametitle{Goals}
    \textbf{1} Develop a subprogram to output a 2-to-1 Euler tour of a graph.
    \begin{itemize}
        \item Takes as input a number of vertices, and an edge set between those vertices.
        \item Outputs an ordering of the vertices that begins at the initial vertex, visits each edge twice, and ends at the initial vertex.
    \end{itemize}
\end{frame}

\begin{frame}
    \frametitle{Goals}
    \textbf{2} Develop a subprogram to output a sequence of vertex sets.
    \begin{itemize}
        \item Inputs a set of points in the plane
        \item Outputs an increasing sequence vertex sets ($X_0 \subseteq X_1 \subseteq X_2 \subseteq \cdots$), given criteria determined in the paper.
    \end{itemize}
\end{frame}

\begin{frame}
    \frametitle{Goals}
    \textbf{3} Develop a subprogram to calculate ``local flatness"
    \begin{itemize}
        \item This program outputs the local flatness of the vertex set $V$ around $v$.
        \item Inputs a vertex set $V$, a vertex $v \in V$.
        \item Inspect a ball around $v$, find a ``line of best fit" for the points in the ball, and output a value corresponding to how well the line fits the points 
    \end{itemize}
\end{frame}

\begin{frame}
    \frametitle{Progress}

    Organization and build system
    \begin{itemize}
        \item igraph included in as a submodule to the repository
        \item build script to build igraph on new system
        \item CMake to build the project
        \item build script to handle calling cmake for the project with the correct library dictories
    \end{itemize}
\end{frame}

\begin{frame}
    \frametitle{Progress}

    Goal 1
    \begin{itemize}
        \item Used standard template library since igraph was not needed for this subprogram
        \item Class structure consists of an unordered map with an integer representing the vertex number as the key and a vector of pairs of integers for the edges. Each pair in the vector has two integers: the first represents the other vertex the vertex is connected to, and the second is how many times this edge has been traversed.
        \item Function that searches for the vertex numbered one and the helper function starting there.
        \item The helper function recursively calls itself after iterating through the adjacency vector for a given node, checking each edge to see if it's been traversed less than twice. At the first available node, it increments the edge both ways, then breaks off and recurses.
    \end{itemize}

\end{frame}

\begin{frame}
    \frametitle{Progress}

    Goal 3
    \begin{itemize}
        \item The flatness function takes in a set of points in $\mathbb{R}^2$, a vertex $v$ as an index into the set of points, a value for $\epsilon$, and a value $k$, from which the radius of the ball is computed. Returns the flatness.
        \item Point set class which handles finding intersections between the ball and the set of points
        \item Needs testing
    \end{itemize}
\end{frame}

\end{document}